\begin{table}[H]
	\begin{tabular}{ | p{0.22\textwidth} | p{0.68\textwidth} |} \hline
		\rowcolor{gray!50}
		%Titelzeile
			\textbf{ID} & \textbf{S1}\\ \hline
		%Zeile
			\textbf{Bezeichnung} & 
            Test des Gesamtsystems ohne UI.\\ \hline
		%Zeile
			\textbf{Beschreibung} & 
            Der \documenttitle\ wird anhand der Testuhr als Ganzes überprüft.\\ \hline
		%Zeile
			\textbf{Akteure} &
            Entwickler\\ \hline
		%Zeile
			\textbf{Vorbedingungen} 
            & \tabitem Netzwerkverbindung vom \documenttitle\ zu einem Test-PC oder Notebook.\\
            & \tabitem Laufende oder zumindest funktionsfähige Testuhr. \\
            & \tabitem Der \documenttitle\ muss an die Stromversorgung angeschlossen sein. \\
            & \tabitem SSH Verbindung zum \documenttitle\ vorhanden.\\ 
            & \tabitem sqlite3 starten mit der Datenbank \glqq clock\_analizer\grqq.\\ \hline
		%Zeile
			\textbf{Ergebnis}
            & \tabitem Das \hwb\ ermittelt regelmässig neue Werte und sendet diese an das \rpi. \\
			& \tabitem Die Datenbank wird mit neuen Werten befüllt.\\ \hline
		%Zeile
			\textbf{Ergebnis bei Fehler}
            & \tabitem Fehlerhafte oder fehlende Werte in der Datenbank.\\
			& \tabitem Die Sensorwerte stimmen nicht mit den Erwartungen überein.\\ \hline
		%Zeile
			\textbf{Ablauf}
            & 1. System gemäss Vorbedingungen vorbereiten.\\
            & 2. Testuhr in Betrieb nehmen, wenn noch nicht erfolgt.\\
			& 2. Ergebnisse über einen Zeitraum von 15min - 1h oder nach Bedarf beobachten\\
			& 3. Ergebnisse auswerten\\ \hline
        %Zeile
			\textbf{Testdaten} &
            Keine benötigt.\\ \hline
        %Untere Abgrenzung
	\end{tabular}
	\caption{Systemtest 1 - Test des Gesamtsystems ohne UI}
	\label{tab:inttest1}
\end{table}