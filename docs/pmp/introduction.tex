%!TEX root = PMP_ClockPendulumAnalyzer.tex
\section{Einführung}
    Hier wird findet man einen kurzen Überblick auf das vorliegende Dokument.
	\subsection{Zweck des Dokuments}
    Das Dokument dient als Report für den Bereich des Projektmanagements des \documenttitle\ PAWI Projekt der Hochschule Luzern.
	\subsection{Zielpublikum}
    Das Dokument richtet sich an den Auftraggeber und verantwortliche Dozenten.
	\subsection{Versionierung}
		\begin{table}[h]
			\centering
			\begin{tabularx}{\textwidth}{|c|c|X|}
			\hline
			\rowcolor{shadecolor}\textbf{Version} & \textbf{Datum} & \textbf{Kommentar}\\ \hline
            V1.4 & 17.11.2017 & Sprint 3 review hinzugefügt\\
            V1.3 & 02.11.2017 & Sprint 2 review hinzugefügt\\
            V1.2 & 19.10.2017 & sprint 1 review hinzugefügt\\
            V1.1 & 29.09.2017 & organisation, rahmenplan und projektrisiken hinzugefügt\\
			V1.0 & 28.09.2017 & initiale datei \\ \hline
			\end{tabularx}
		\end{table}
	\subsection{Glossar}
		\begin{description}
			\item[RTC]- Real Time Clock
		\end{description}