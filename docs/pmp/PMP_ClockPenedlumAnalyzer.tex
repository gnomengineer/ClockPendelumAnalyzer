\documentclass[a4paper, 10pt, fleqn]{article}
\usepackage{../custom}
\usepackage{../pageformatting}

\usepackage[ngerman]{babel}
%mathe packages
\usepackage{amsmath}

\graphicspath{{pdf/}{images/}}

%============PAGE PROPERTIES=============
\newcommand{\revisiondate}{\today}
\newcommand{\documenttitle}{Clock Pendelum Analyzer} % used for title in title and subtitle pages
\newcommand{\authors}{Tobias Kreienbühl \& Daniel Föhn} %used for title page only
\newcommand{\subauthors}{im Auftrag der Hochschule Luzern} %used for title page only
\newcommand{\subtitle}{PMP}%usedfortitleandsubtitlepages
\newcommand{\documentdesc}{Ein Projekt Management Plan für \documenttitle}


\begin{document}
	\begin{titlepage}
		\titleGM
		\thispagestyle{empty}
	\end{titlepage}
	
	\tableofcontents
	\listoffigures
	\listoftables
	
	\clearpage
	\section{Einführung}
		\subsection{Zweck des Dokuments}
		\subsection{Zielpublikum}
		\subsection{Versionierung}
			\begin{table}[h]
				\centering
				\begin{tabularx}{\textwidth}{|c|c|X|}
				\hline
				\rowcolor{shadecolor}\textbf{Version} & \textbf{Datum} & \textbf{Kommentar}\\ \hline
				V1.0 & \today & initial file \\ \hline
				\end{tabularx}
			\end{table}
		\subsection{Glossar}
			\begin{description}
				\item[Abkürzung]- Erklärung
			\end{description}

\subtitlepage{Projektmanagement}
	\section{Projektorganisation}
		\textit{Wer ist involviert und wer ist zuständig für was}
	\section{Projektrahmenplan}
		\textit{Was sind die Meilensteine und Projektziele}
	\section{Risikomanagement}
		\textit{Was sind die Projektrisiken}
	\section{Zeitplanung}
		\textit{Zeitplan der Sprints und Meilensteine}
	\section{Test}
		\subsection{Testumgebung}
			\textit{in welcher Umgebung wurden die Tests gemacht}
		\subsection{Testfälle}
			\textit{Was wird durch das Testen abgedeckt}
			\subsubsection{Unit Tests}
			\subsubsection{Blackbox Tests}

\clearpage
\thispagestyle{empty}
	\section*{Anhang}
\end{document}
