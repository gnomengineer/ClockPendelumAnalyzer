\documentclass[a4paper, 10pt, fleqn]{article}
\usepackage{../custom}
\usepackage{../pageformatting}

\usepackage[ngerman]{babel}
%mathe packages
\usepackage{amsmath}
\usepackage{booktabs}
\newcommand{\tabitem}{~~\llap{\textbullet}~~}

\graphicspath{{pdf/}{../images/}}

%============PAGE PROPERTIES=============
\newcommand{\revisiondate}{\today}
\newcommand{\documenttitle}{Clock Pendelum Analyzer} % used for title in title and subtitle pages
\newcommand{\authors}{Tobias Kreienbühl \& Daniel Föhn} %used for title page only
\newcommand{\subauthors}{im Auftrag der Hochschule Luzern} %used for title page only
\newcommand{\subtitle}{PMP}%usedfortitleandsubtitlepages
\newcommand{\documentdesc}{Ein Projekt Management Plan für \documenttitle}


\begin{document}
	\begin{titlepage}
		\titleGM
		\thispagestyle{empty}
	\end{titlepage}
	
	\tableofcontents
	\listoffigures
	\listoftables
	
	\clearpage
	\section{Einführung}
		\subsection{Zweck des Dokuments}
		\subsection{Zielpublikum}
		\subsection{Versionierung}
			\begin{table}[h]
				\centering
				\begin{tabularx}{\textwidth}{|c|c|X|}
				\hline
				\rowcolor{shadecolor}\textbf{Version} & \textbf{Datum} & \textbf{Kommentar}\\ \hline
				V1.0 & 28.09.2017 & initiale datei \\ \hline
                V1.1 & 29.09.2017 & rahmenplan und projektrisiken\\
				\end{tabularx}
			\end{table}
		\subsection{Glossar}
			\begin{description}
				\item[Abkürzung]- Erklärung
			\end{description}

\subtitlepage{Projektmanagement}
	\section{Projektorganisation}
        \begin{figure}[H]
            \centering
            \includegraphics[width=.5\textwidth]{organisation.png}
            \caption{einfache Projektorganisationsstruktur}
        \end{figure}
		\textbf{Entwickler:} Tobias Kreienbühl
        \begin{itemize}
            \item Projektplanung
            \item Entwicklung der Software
            \item Entwicklung der Mechanik
            \item Mathematische Umsetzung
        \end{itemize}
        \vspace{.5cm}
        \textbf{Entwickler:} Daniel Föhn
        \begin{itemize}
            \item Projektplanung
            \item Entwicklung der Software
            \item Entwicklung der Elektronik
            \item Aufbau der Environment
        \end{itemize}
        \vspace{.5cm}
        \textbf{Auftraggeber:} Josef Bürgler
        \begin{itemize}
            \item Anforderungen abnehmen
            \item 
        \end{itemize}
    
    \clearpage
	\section{Projektrahmenplan}
        \begin{figure}[H]
            \centering
            \includegraphics[width=\textwidth]{../../rahmenplan.png}
            \caption{Rahmenplan mit Phasen, Meilensteine und Sprints}
        \end{figure}
        \begin{tabularx}{\textwidth}{lll}
            \textbf{MS1} & Zeitpunkt: & Freitag 6.10.\\
            & Artefakte: & \tabitem PMP\\
            & & \tabitem Entwurf des Grobkonzepts\\
            & Ergebnisse: & \tabitem definierte Vorgehensart\\
            & & \tabitem Rahmenplanung\\
            & & \tabitem Vision (Scope, Ziele etc) im Grobkonzept\\
            \textbf{MS2} & Zeitpunkt: & Donnerstag 16.11.2017\\
            & Artefakte & \tabitem Prototyp 1\\
            & Ergebnisse: & \tabitem lauffähiger 1. Prototyp\\
            & & \tabitem 80\% der Sys Spec\\
            \textbf{MS3} & Zeitpunkt: & Donnerstag 14.12.2017\\
            & Artefakte & \tabitem PMP \\
            & & \tabitem SysSpec \\
            & & \tabitem Arbeitsjournal \\
            & & \tabitem Prototyp 2\\
            & Ergebnisse: & \tabitem lauffähiger 2. Prototyp\\
            & & \tabitem fertige System Spezifikation (Projektreport)\\
            & & \tabitem fertiger PMP\\
        \end{tabularx}
    \clearpage
	\section{Zeitplanung}
        Das Projekt wird mit einer agilen Zeitplanung mit Hilfe von Sprints durchgeführt. Die Sprints dauern jeweils 2 Wochen
        \begin{figure}[H]
            \centering
            \includegraphics[width=.5\textwidth]{sprint_overview.png}
        \end{figure}

    \clearpage
            
	\section{Risikomanagement}
		\textit{Was sind die Projektrisiken}

    \clearpage
    %!TEX root = PMP_ClockPendulumAnalyzer.tex
\section{Test}
		\subsection{Testumgebung}
        In diesem Kapitel ist die ganze Testumgebung erläutert, damit exakte und nachvollziehbare Test gemacht werden können.
        \subsubsection{Aufstellung}
        Der Pendulum Analyzer wird mittels einer normalen Wand-Pendeluhr getestet. Dazu wird die Uhr in einer dafür hergestellten Verschalung aufgehängt.
        \begin{figure}[H]
            \centering
            \includegraphics[width=.5\textwidth]{verschalung.png}
            \caption{Verschalung für die Pendeluhr}
        \end{figure}

        \noindent Die Uhr hat eine Fläche auf der das Gerät platziert werden kann. Es wird daher keine zusätzliche Montage für die Sensoren gebaut.
        
        \subsubsection{Software Komponenten}

		\subsection{Testfälle}
			\textit{Was wird durch das Testen abgedeckt}
			\subsubsection{Unit Tests}
			\subsubsection{Blackbox Tests}
	

\clearpage
\thispagestyle{empty}
	\section*{Anhang}
\end{document}
