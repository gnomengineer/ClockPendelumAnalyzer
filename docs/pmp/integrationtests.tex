%\newcommand{\tempitem}{\labelitemi}
\renewcommand{\labelitemi}{-}

\begin{table}[H]
	\begin{tabularx}{\textwidth}{ | p{0.22\textwidth} | p{0.68\textwidth} |} \hline
		\rowcolor{gray!50}
		%Titelzeile
			\textbf{ID} & \textbf{I1}\\ \hline
		%Zeile
			\textbf{Bezeichnung} & 
            Ermitteln Referenzfrequenz ohne GPS Verbindung.\\ \hline
		%Zeile
			\textbf{Beschreibung} & 
            Die Referenzfrequenz wird mittels FIX-Pin des GPS-Moduls ermittelt.\\ \hline
		%Zeile
			\textbf{Akteure} &
            Entwickler\\ \hline
		%Zeile
			\textbf{Vorbedingungen} &
            \begin{itemize}
                \item USB-Verbindung von \hwb\ zu einem Test-PC oder Notebook.
                \item Optische Überprüfung des laufenden \hwb.
                \item Die FIX-LED des GPS-Moduls auf dem \hwb\ blinkt im Sekundentakt.
                \item Konsolenprogramm zum Empfang von UART-RS232 Signalen gestartet.
            \end{itemize}\\ \hline
		%Zeile
			\textbf{Ergebnis} &        
			\begin{itemize}
				\item Die Referenzfrequenzen werden jede Sekunde aufgelistet.
				\item Die Streuung zwischen den Werten entspricht der Streuung gemäss Systemspezifikation (Kapitel ).
			\end{itemize}\\ \hline
		%Zeile
			\textbf{Ergebnis bei Fehler} &
			\begin{itemize}
				\item Fehlerhafte oder fehlende Werte in der Anzeige.
				\item Anzeigen auf dem \hwb\ verhalten sich nicht erwartungsgemäss.
			\end{itemize}\\ \hline
		%Zeile
			\textbf{Ablauf} &
			\begin{enumerate}
				\item USB-Verbindung erstellen.
				\item Terminal-Programm starten (z.B. Minicom)
				\item Ergebnis prüfen
			\end{enumerate}\\ \hline
        %Zeile
			\textbf{Testdaten} &
            Keine benötigt.\\ \hline
        %Untere Abgrenzung
	\end{tabularx}
	\caption{Integrationstest 1}
	\label{tab:inttest1}
\end{table}

\renewcommand{\labelitemi}{$\bullet$}