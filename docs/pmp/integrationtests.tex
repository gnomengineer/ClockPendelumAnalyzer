%\newcommand{\tempitem}{\labelitemi}
\renewcommand{\labelitemi}{-}

\begin{table}[H]
	\begin{tabularx}{\textwidth}{ | p{0.22\textwidth} | p{0.68\textwidth} |} \hline
		\rowcolor{gray!50}
		%Titelzeile
			\textbf{ID} & \textbf{I1}\\ \hline
		%Zeile
			\textbf{Bezeichnung} & 
            Ermitteln Referenzfrequenz ohne GPS Verbindung.\\ \hline
		%Zeile
			\textbf{Beschreibung} & 
            Die Referenzfrequenz wird mittels FIX-Pin des GPS-Moduls ermittelt.\\ \hline
		%Zeile
			\textbf{Akteure} &
            Entwickler\\ \hline
		%Zeile
			\textbf{Vorbedingungen} &
            \begin{itemize}
                \item USB-Verbindung von \hwb\ zu einem Test-PC oder Notebook.
                \item Optische Überprüfung des laufenden \hwb.
                \item Die FIX-LED des GPS-Moduls auf dem \hwb\ blinkt im Sekundentakt.
                \item Konsolenprogramm zum Empfang von UART-RS232 Signalen gestartet.
            \end{itemize}\\ \hline
		%Zeile
			\textbf{Ergebnis} &        
			\begin{itemize}
				\item Die Referenzfrequenzen werden jede Sekunde aufgelistet.
				\item Die Streuung zwischen den Werten entspricht der Streuung gemäss Systemspezifikation (Kapitel ).
			\end{itemize}\\ \hline
		%Zeile
			\textbf{Ergebnis bei Fehler} &
			\begin{itemize}
				\item Fehlerhafte oder fehlende Werte in der Anzeige.
				\item Anzeigen auf dem \hwb\ verhalten sich nicht erwartungsgemäss.
			\end{itemize}\\ \hline
		%Zeile
			\textbf{Ablauf} &
			\begin{enumerate}
				\item USB-Verbindung erstellen.
				\item Terminal-Programm starten (z.B. Minicom)
				\item Ergebnis prüfen
			\end{enumerate}\\ \hline
        %Zeile
			\textbf{Testdaten} &
            Keine benötigt.\\ \hline
        %Untere Abgrenzung
	\end{tabularx}
	\caption{Integrationstest 1 - Ermitteln Referenzfrequenz ohne GPS}
	\label{tab:inttest1}
\end{table}

\begin{table}[H]
	\begin{tabularx}{\textwidth}{ | p{0.22\textwidth} | p{0.68\textwidth} |} \hline
		\rowcolor{gray!50}
		%Titelzeile
			\textbf{ID} & \textbf{I2}\\ \hline
		%Zeile
			\textbf{Bezeichnung} & 
            Ermitteln Referenzfrequenz mit GPS Verbindung.\\ \hline
		%Zeile
			\textbf{Beschreibung} & 
            Die Referenzfrequenz wird mittels PPS-Pin des GPS-Moduls ermittelt.\\ \hline
		%Zeile
			\textbf{Akteure} &
            Entwickler\\ \hline
		%Zeile
			\textbf{Vorbedingungen} &
            \begin{itemize}
                \item USB-Verbindung von \hwb\ zu einem Test-PC oder Notebook.
                \item Optische Überprüfung des laufenden \hwb.
                \item Die FIX-LED des GPS-Moduls auf dem \hwb\ blinkt alle 15 Sekunden.
                \item Konsolenprogramm zum Empfang von UART-RS232 Signalen gestartet.
            \end{itemize}\\ \hline
		%Zeile
			\textbf{Ergebnis} &        
			\begin{itemize}
				\item Die Referenzfrequenzen werden jede Sekunde aufgelistet.
				\item Die Streuung zwischen den Werten entspricht der Streuung gemäss Systemspezifikation (Kapitel ).
			\end{itemize}\\ \hline
		%Zeile
			\textbf{Ergebnis bei Fehler} &
			\begin{itemize}
				\item Fehlerhafte oder fehlende Werte in der Anzeige.
				\item Anzeigen auf dem \hwb\ verhalten sich nicht erwartungsgemäss.
			\end{itemize}\\ \hline
		%Zeile
			\textbf{Ablauf} &
			\begin{enumerate}
				\item USB-Verbindung erstellen.
				\item Terminal-Programm starten (z.B. Minicom)
				\item Ergebnis prüfen
			\end{enumerate}\\ \hline
        %Zeile
			\textbf{Testdaten} &
            Keine benötigt.\\ \hline
        %Untere Abgrenzung
	\end{tabularx}
	\caption{Integrationstest 1 - Ermitteln Referenzfrequenz ohne GPS}
	\label{tab:inttest2}
\end{table}

\begin{table}[H]
    \begin{tabularx}{\textwidth}{ | p{0.22\textwidth} | p{0.68\textwidth} |} \hline
        \rowcolor{gray!50}
        %Titelzeile
        \textbf{ID} & \textbf{I3}\\ \hline
        %Zeile
        \textbf{Bezeichnung} & 
        Speichern von Datensätzen.\\ \hline
        %Zeile
        \textbf{Beschreibung} & 
        Datensätze werden korrekt in der SQLite Datenbank abgespeichert.\\ \hline
        %Zeile
        \textbf{Akteure} &
        Entwickler\\ \hline
        %Zeile
        \textbf{Vorbedingungen} &
        \begin{itemize}
            \item SQLite installiert.
            \item Konsolenprogramm zum Lesen der SQLite Datenbank gestartet.
        \end{itemize}\\ \hline
        %Zeile
        \textbf{Ergebnis} &        
        \begin{itemize}
            \item Es sind neue Datensätze vorhanden.
            \item Korrektes Speichern wird vom System gemeldet.
        \end{itemize}\\ \hline
        %Zeile
        \textbf{Ergebnis bei Fehler} &
        \begin{itemize}
            \item Keine neuen Datensätze vorhanden.
            \item Fehler wird vom System gemeldet.
        \end{itemize}\\ \hline
        %Zeile
        \textbf{Ablauf} &
        \begin{enumerate}
            \item Programm starten.
            \item Warten bis Bestätigung des Speicherns erscheint.
            \item Datenbank in Konsole lesen.
            \item Ergebnis prüfen.
        \end{enumerate}\\ \hline
        %Zeile
        \textbf{Testdaten} &
        Testdaten werden mittels Software erstellt.\\ \hline
        %Untere Abgrenzung
    \end{tabularx}
    \caption{Integrationstest 2 - Speichern von Datensätzen}
    \label{tab:inttest3}
\end{table}

\begin{table}[H]
    \begin{tabularx}{\textwidth}{ | p{0.22\textwidth} | p{0.68\textwidth} |} \hline
        \rowcolor{gray!50}
        %Titelzeile
        \textbf{ID} & \textbf{I4}\\ \hline
        %Zeile
        \textbf{Bezeichnung} & 
        Abfragen der Daten über REST.\\ \hline
        %Zeile
        \textbf{Beschreibung} & 
        Daten können über die REST Schnittstelle korrekt empfangen werden.\\ \hline
        %Zeile
        \textbf{Akteure} &
        Entwickler\\ \hline
        %Zeile
        \textbf{Vorbedingungen} &
        \begin{itemize}
            \item Es existieren Daten zum lesen.
            \item Programm ist gestartet
        \end{itemize}\\ \hline
        %Zeile
        \textbf{Ergebnis} &        
        \begin{itemize}
            \item Es wird eine einfache Webseite angezeigt mit allen Daten seit Messbeginn.
        \end{itemize}\\ \hline
        %Zeile
        \textbf{Ergebnis bei Fehler} &
        \begin{itemize}
            \item Es wird eine Fehlerseite angezeigt.
        \end{itemize}\\ \hline
        %Zeile
        \textbf{Ablauf} &
        \begin{enumerate}
            \item Browser öffnen
            \item URL ''http://192.168.192.75?date'' eingeben
            \item Ergebnis prüfen.
        \end{enumerate}\\ \hline
        %Zeile
        \textbf{Testdaten} &
        Testdaten werden mittels Software erstellt.\\ \hline
        %Untere Abgrenzung
    \end{tabularx}
    \caption{Integrationstest 3 - Abrufen von Daten mit REST}
    \label{tab:inttest4}
\end{table}

\renewcommand{\labelitemi}{$\bullet$}