\renewcommand{\labelitemi}{-}

\begin{table}[H]
	\begin{tabular}{ | p{0.22\textwidth} | p{0.68\textwidth} |} \hline
		\rowcolor{gray!50}
		%Titelzeile
			\textbf{ID} & \textbf{I1}\\ \hline
		%Zeile
			\textbf{Bezeichnung} & 
            Ermitteln Referenzfrequenz ohne GPS Verbindung.\\ \hline
		%Zeile
			\textbf{Beschreibung} & 
            Die Referenzfrequenz wird mittels FIX-Pin des GPS-Moduls ermittelt.\\ \hline
		%Zeile
			\textbf{Akteure} &
            Entwickler\\ \hline
		%Zeile
			\textbf{Vorbedingungen} 
            & \tabitem USB-Verbindung von \hwb\ zu einem Test-PC oder Notebook.\\
            & \tabitem Optische Überprüfung des laufenden \hwb. \\
            & \tabitem Die FIX-LED des GPS-Moduls auf dem \hwb\ blinkt im Sekundentakt. \\
            & \tabitem Konsolenprogramm zum Empfang von UART-RS232 Signalen gestartet.\\ \hline
		%Zeile
			\textbf{Ergebnis}
            &\tabitem Die Referenzfrequenzen werden jede Sekunde aufgelistet.\\
			&\tabitem Die Streuung zwischen den Werten entspricht der Streuung gemäss Systemspezifikation (Kapitel ).\\ \hline
		%Zeile
			\textbf{Ergebnis bei Fehler}
            & \tabitem Fehlerhafte oder fehlende Werte in der Anzeige.\\
			& \tabitem Anzeigen auf dem \hwb\ verhalten sich nicht erwartungsgemäss.\\ \hline
		%Zeile
			\textbf{Ablauf}
            & 1. USB-Verbindung erstellen.\\
			& 2. Terminal-Programm starten (z.B. Minicom)\\
			& 3. Ergebnis prüfen\\ \hline
        %Zeile
			\textbf{Testdaten} &
            Keine benötigt.\\ \hline
        %Untere Abgrenzung
	\end{tabular}
	\caption{Integrationstest 1 - Ermitteln Referenzfrequenz ohne GPS}
	\label{tab:inttest1}
\end{table}

\begin{table}[H]
	\begin{tabular}{ | p{0.22\textwidth} | p{0.68\textwidth} |} \hline
		\rowcolor{gray!50}
		%Titelzeile
			\textbf{ID} & \textbf{I2}\\ \hline
		%Zeile
			\textbf{Bezeichnung} & 
            Ermitteln Referenzfrequenz mit GPS Verbindung.\\ \hline
		%Zeile
			\textbf{Beschreibung} & 
            Die Referenzfrequenz wird mittels PPS-Pin des GPS-Moduls ermittelt.\\ \hline
		%Zeile
			\textbf{Akteure} &
            Entwickler\\ \hline
		%Zeile
			\textbf{Vorbedingungen}
            & \tabitem USB-Verbindung von \hwb\ zu einem Test-PC oder Notebook.\\
            & \tabitem Optische Überprüfung des laufenden \hwb. \\
            & \tabitem Die FIX-LED des GPS-Moduls auf dem \hwb\ blinkt alle 15 Sekunden. \\
            & \tabitem Konsolenprogramm zum Empfang von UART-RS232 Signalen gestartet.\\ \hline
		%Zeile
			\textbf{Ergebnis}        
			& \tabitem Die Referenzfrequenzen werden jede Sekunde aufgelistet.\\
			& \tabitem Die Streuung zwischen den Werten entspricht der Streuung gemäss Systemspezifikation (Kapitel ).\\ \hline
		%Zeile
			\textbf{Ergebnis bei Fehler}
			& \tabitem Fehlerhafte oder fehlende Werte in der Anzeige.\\
			& \tabitem Anzeigen auf dem \hwb\ verhalten sich nicht erwartungsgemäss.\\ \hline
		%Zeile
			\textbf{Ablauf}
			& 1. USB-Verbindung erstellen.\\
			& 2. Terminal-Programm starten (z.B. Minicom)\\
			& 3. Ergebnis prüfen\\ \hline
        %Zeile
			\textbf{Testdaten} &
            Keine benötigt.\\ \hline
        %Untere Abgrenzung
	\end{tabular}
	\caption{Integrationstest 1 - Ermitteln Referenzfrequenz ohne GPS}
	\label{tab:inttest2}
\end{table}

\begin{table}[H]
    \begin{tabular}{ | p{0.22\textwidth} | p{0.68\textwidth} |} \hline
        \rowcolor{gray!50}
        %Titelzeile
        \textbf{ID} & \textbf{I3}\\ \hline
        %Zeile
        \textbf{Bezeichnung} & 
        Speichern von Datensätzen.\\ \hline
        %Zeile
        \textbf{Beschreibung} & 
        Datensätze werden korrekt in der SQLite Datenbank abgespeichert.\\ \hline
        %Zeile
        \textbf{Akteure} &
        Entwickler\\ \hline
        %Zeile
        \textbf{Vorbedingungen}
        & \tabitem SQLite installiert.\\
        & \tabitem Konsolenprogramm zum Lesen der SQLite Datenbank gestartet.\\ \hline
        %Zeile
        \textbf{Ergebnis}
        & \tabitem Es sind neue Datensätze vorhanden.\\
        & \tabitem Korrektes Speichern wird vom System gemeldet.\\ \hline
        %Zeile
        \textbf{Ergebnis bei Fehler}
        & \tabitem Keine neuen Datensätze vorhanden.\\
        & \tabitem Fehler wird vom System gemeldet.\\ \hline
        %Zeile
        \textbf{Ablauf}
        & 1. Programm starten.\\
        & 2. Warten bis Bestätigung des Speicherns erscheint.\\
        & 3. Datenbank in Konsole lesen.\\
        & 4. Ergebnis prüfen.\\ \hline
        %Zeile
        \textbf{Testdaten} &
        Testdaten werden mittels Software erstellt.\\ \hline
        %Untere Abgrenzung
    \end{tabular}
    \caption{Integrationstest 2 - Speichern von Datensätzen}
    \label{tab:inttest3}
\end{table}

\begin{table}[H]
    \begin{tabular}{ | p{0.22\textwidth} | p{0.68\textwidth} |} \hline
        \rowcolor{gray!50}
        %Titelzeile
        \textbf{ID} & \textbf{I4}\\ \hline
        %Zeile
        \textbf{Bezeichnung} & 
        Abfragen der Daten über REST.\\ \hline
        %Zeile
        \textbf{Beschreibung} & 
        Daten können über die REST Schnittstelle korrekt empfangen werden.\\ \hline
        %Zeile
        \textbf{Akteure} &
        Entwickler\\ \hline
        %Zeile
        \textbf{Vorbedingungen}
        & \tabitem Es existieren Daten zum lesen.\\
        & \tabitem Programm ist gestartet\\ \hline
        %Zeile
        \textbf{Ergebnis}
        & \tabitem Es wird eine einfache Webseite angezeigt mit allen Daten seit Messbeginn.\\ \hline
        %Zeile
        \textbf{Ergebnis bei Fehler}
        & \tabitem Es wird eine Fehlerseite angezeigt.\\ \hline
        %Zeile
        \textbf{Ablauf}
        & 1. Browser öffnen\\
        & 2. URL ''http://192.168.192.75?date'' eingeben\\
        & 3. Ergebnis prüfen.\\ \hline
        %Zeile
        \textbf{Testdaten} &
        Testdaten werden mittels Software erstellt.\\ \hline
        %Untere Abgrenzung
    \end{tabular}
    \caption{Integrationstest 3 - Abrufen von Daten mit REST}
    \label{tab:inttest4}
\end{table}

\renewcommand{\labelitemi}{$\bullet$}