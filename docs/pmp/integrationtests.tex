\begin{table}[H]
	\begin{tabular}{ | p{0.22\textwidth} | p{0.68\textwidth} |} \hline
		\rowcolor{gray!50}
		%Titelzeile
			\textbf{ID}          &
			\begin{tabular}{l}
				\textbf{I1}
			\end{tabular}
			\\ \hline
		%Zeile
			\textbf{Bezeichnung}			 &
			\begin{tabular}{l}
				Ermitteln Referenzfrequenz ohne GPS Verbindung.
			\end{tabular}
			\\ \hline
		%Zeile
			\textbf{Beschreibung}   		 &
			\begin{tabular}{l}
				Die Referenzfrequenz wird mittels FIX-Pin des GPS-Moduls ermittelt.
			\end{tabular}
		\\ \hline
		%Zeile
			\textbf{Akteure}              & 
			\begin{tabular}{l}
				Entwickler
			\end{tabular}
		\\ \hline
		%Zeile
			\textbf{Vorbedingungen}       &
			\begin{tabular}{l}
				USB-Verbindung von \hwb zu einem Test-PC oder Notebook.\\
				Optische Überprüfung des laufenden \hwb.\\
				Die FIX-LED des GPS-Moduls auf dem \hwb blinkt im Sekundentakt. \\
				Konsolenprogramm zum Empfang von UART-RS232 Signalen gestartet.
			\end{tabular}		
		\\ \hline
		%Zeile
			\textbf{Ergebnis}             &        
			\begin{tabular}{l}
				Die Referenzfrequenzen werden jede Sekunde aufgelistet. \\
				Die Streuung zwischen den Werten entspricht der Streuung gemäss Systemspezifikation (Kapitel ).
			\end{tabular}
		\\ \hline
		%Zeile
			\textbf{Ergebnis bei Fehler}  &
			\begin{tabular}{l}
				- Fehlerhafte oder fehlende Werte in der Anzeige\\
				- Anzeigen auf dem \hwb verhalten sich nicht erwartungsgemäss.
			\end{tabular}
		\\ \hline
		%Zeile
			\textbf{Ablauf}				 &
			\begin{enumerate}
				\item USB-Verbindung erstellen.
				\item Terminal-Programm starten (z.B. Minicom)
				\item Ergebnis prüfen
			\end{enumerate}
		\\ \hline
			\textbf{Testdaten}            &
			\begin{tabular}{l}
				Keine benötigt.
			\end{tabular}
		\\ \hline	%Untere Abgrenzung
	\end{tabular}
	\caption{Integrationstest 1}
	\label{tab:inttest1}
\end{table}