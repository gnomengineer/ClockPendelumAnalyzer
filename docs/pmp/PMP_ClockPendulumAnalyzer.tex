\documentclass[a4paper, 10pt, fleqn]{article}
\usepackage{../custom}
\usepackage{../pageformatting}

\usepackage[ngerman]{babel}
%mathe packages
\usepackage{amsmath}
\usepackage{booktabs}
\newcommand{\tabitem}{~~\llap{\textbullet}~~}

\graphicspath{{pdf/}{../images/}}

%============PAGE PROPERTIES=============
\newcommand{\revisiondate}{\today}
\newcommand{\documenttitle}{Clock Pendulum Analyzer} % used for title in title and subtitle pages
\newcommand{\authors}{Tobias Kreienbühl \& Daniel Föhn} %used for title page only
\newcommand{\subauthors}{im Auftrag der Hochschule Luzern} %used for title page only
\newcommand{\subtitle}{PMP}%usedfortitleandsubtitlepages
\newcommand{\documentdesc}{Ein Projekt Management Plan für \documenttitle}


\begin{document}
	\begin{titlepage}
		\titleGM
		\thispagestyle{empty}
	\end{titlepage}
	
	\tableofcontents
	\listoffigures
	\listoftables
	
	\clearpage
	\section{Einführung}
        Hier wird findet man einen kurzen Überblick auf das vorliegende Dokument.
		\subsection{Zweck des Dokuments}
        Das Dokument dient als Report für den Bereich des Projektmanagements des \documenttitle\ PAWI Projekt der Hochschule Luzern.
		\subsection{Zielpublikum}
        Das Dokument richtet sich an den Auftraggeber und verantwortliche Dozenten.
		\subsection{Versionierung}
			\begin{table}[h]
				\centering
				\begin{tabularx}{\textwidth}{|c|c|X|}
				\hline
				\rowcolor{shadecolor}\textbf{Version} & \textbf{Datum} & \textbf{Kommentar}\\ \hline
                V1.2 & 19.10.2017 & sprint 1 review hinzugefügt\\
                V1.1 & 29.09.2017 & organisation, rahmenplan und projektrisiken hinzugefügt\\
				V1.0 & 28.09.2017 & initiale datei \\ \hline
				\end{tabularx}
			\end{table}
		\subsection{Glossar}
			\begin{description}
				\item[RTC]- Real Time Clock
			\end{description}

\subtitlepage{Projektmanagement}
	\section{Projektorganisation}
        Das Projekt besteht aus 2 Entwicklern und einem Auftraggeber der als Ansprechperson gilt. Dadurch wird die Projektorganisation möglichst leicht gehalten. Die Verantwortung wird gleichmässig auf die beiden Entwickler aufgeteilt.
        \begin{figure}[H]
            \centering
            \includegraphics[width=.5\textwidth]{organisation.png}
            \caption{einfache Projektorganisationsstruktur}
        \end{figure}
		\textbf{Entwickler:} Tobias Kreienbühl
        \begin{itemize}
            \item Projektplanung
            \item Entwicklung der Software
            \item Entwicklung der Mechanik
            \item Mathematische Umsetzung
        \end{itemize}
        \vspace{.5cm}
        \textbf{Entwickler:} Daniel Föhn
        \begin{itemize}
            \item Projektplanung
            \item Entwicklung der Software
            \item Entwicklung der Elektronik
            \item Aufbau der Environment
        \end{itemize}
        \vspace{.5cm}
        \textbf{Auftraggeber:} Josef Bürgler
        \begin{itemize}
            \item Anforderungen abnehmen
            \item 
        \end{itemize}
    
    \clearpage
	\section{Projektrahmenplan}
        In diesem Kapitel werden die Meilensteine und Eckdaten wie Start- und Endzeitpunkt des Projekts festgehalten.
        \begin{figure}[H]
            \centering
            \includegraphics[width=\textwidth]{rahmenplan.png}
            \caption{Rahmenplan mit Phasen, Meilensteine und Sprints}
        \end{figure}
        \begin{tabularx}{\textwidth}{lll}
            \textbf{MS1} & Zeitpunkt: & Freitag 6.10.\\
            & Artefakte: & \tabitem PMP\\
            & & \tabitem Entwurf des Grobkonzepts\\
            & Ergebnisse: & \tabitem definierte Vorgehensart\\
            & & \tabitem Rahmenplanung\\
            & & \tabitem Vision (Scope, Ziele etc) im Grobkonzept\\
            \textbf{MS2} & Zeitpunkt: & Donnerstag 16.11.2017\\
            & Artefakte & \tabitem Prototyp 1\\
            & Ergebnisse: & \tabitem lauffähiger 1. Prototyp\\
            & & \tabitem 80\% der Sys Spec\\
            \textbf{MS3} & Zeitpunkt: & Donnerstag 14.12.2017\\
            & Artefakte & \tabitem PMP \\
            & & \tabitem SysSpec \\
            & & \tabitem Arbeitsjournal \\
            & & \tabitem Prototyp 2\\
            & Ergebnisse: & \tabitem lauffähiger 2. Prototyp\\
            & & \tabitem fertige System Spezifikation (Projektreport)\\
            & & \tabitem fertiger PMP\\
        \end{tabularx}
    \clearpage
	\section{Zeitplanung}
        Das Projekt wird mit einer agilen Zeitplanung in Form von Sprints durchgeführt. Ein Sprint dauert jeweils 2 Wochen.
        \begin{figure}[H]
            \centering
            \includegraphics[width=.5\textwidth]{sprint_overview.png}
        \end{figure}

    \clearpage
            
	\section{Risikomanagement}
		\textit{Was sind die Projektrisiken}

    \clearpage
    %!TEX root = PMP_ClockPendulumAnalyzer.tex
\section{Test}
		\subsection{Testumgebung}
        In diesem Kapitel ist die ganze Testumgebung erläutert, damit exakte und nachvollziehbare Test gemacht werden können.
        \subsubsection{Aufstellung}
        Der Pendulum Analyzer wird mittels einer normalen Wand-Pendeluhr getestet. Dazu wird die Uhr in einer dafür hergestellten Verschalung aufgehängt.
        \begin{figure}[H]
            \centering
            \includegraphics[width=.5\textwidth]{verschalung.png}
            \caption{Verschalung für die Pendeluhr}
        \end{figure}

        \noindent Die Uhr hat eine Fläche auf der das Gerät platziert werden kann. Es wird daher keine zusätzliche Montage für die Sensoren gebaut.
        
        \subsubsection{Software Komponenten}

		\subsection{Testfälle}
			\textit{Was wird durch das Testen abgedeckt}
			\subsubsection{Unit Tests}
			\subsubsection{Blackbox Tests}
	

\clearpage
\thispagestyle{empty}
	\section*{Anhang}
    %!TEX root = PMP_ClockPendulumAnalyzer.tex
\subsection*{Sprint 1}
Die Planung und der Abschluss von Sprint 1 ist in diesem Kapitel aufgeführt.
\subsubsection*{Planung}
Die Planung der Sprints wurde mit dem Programm Taiga.io durchgeführt. Unten ist ein Screenshot zu Beginn von Sprint 1.
    \begin{figure}[H]
        \centering
        \includegraphics[width=.5\textwidth]{sprint1_plan.png}
        \caption{Sprintplanung für Sprint 1}
    \end{figure}
\subsubsection*{Sprintreview}
    Sprint 1 wurde mit zufrieden stellenden Ergebnissen aus verschiedenen einzelnen Tasks beendet. Die involvierten User Stories wurden mit untenstehender Begründung in den nächsten Sprint verschoben.
    \begin{table}[H]
        \centering
        \begin{tabular}{lccp{7cm}}
            \textbf{User Story} &  \textbf{Status} & \textbf{Sprintziel}& \textbf{Begründung}\\\toprule[2pt]
            \#19 Referenz Clock messen & teilweise erledigt & Verschoben & Auf Grund von Wartezeiten\\
            \#20 Sensordaten visualisieren & teilweise erledigt & Verschoben & einzelne Tasks konnten wegen fehlendem Werkzeug nicht komplettiert werden\\
            \#18 Messung Sensoren & teilweise erledigt & Verschoben & Tasks konnten aufgrund Abwesenheit nicht vervollständigt werden\\
        \end{tabular}
        \caption{Status der User Stories aus Sprint 1}
    \end{table}

\clearpage
\subsection*{Sprint 2}
Die Planung und der Abschluss von Sprint 1 ist in diesem Kapitel aufgeführt.
\subsubsection*{Planung}
Der Sprint enthält die gleichen User Stories wie der 1. Sprint, weil alle Pakete in diesen Sprint verschoben wurden.
\begin{figure}[H]
    \centering
    \includegraphics[width=.5\textwidth]{sprint2_plan.png}
    \caption{Sprintplanung für Sprint 2}
\end{figure}
\subsubsection*{Sprintreview}
Aufgrund von zu wenig Wissen über Hardware und Sensorik war der Fortschritt nur schwerfällig. Mit Hilfe eines Logic Analyzer (LA) konnte die RTC und der Sensor überprüft werden. Daraus ergab sich das der Sensor neu gelötet werden muss. Die RTC war in Ordnung. Der Zugriff auf die GPIO wurde ebenfalls umgesetzt. Somit wurden alle Sprintziele erreicht
\begin{table}[H]
    \centering
    \begin{tabular}{lccp{7cm}}
        \textbf{User Story} &  \textbf{Status} & \textbf{Sprintziel}& \textbf{Begründung}\\\toprule[2pt]
        \#19 Referenz Clock messen & erledigt & erreicht & mit LA gemessen\\
        \#20 Sensordaten visualisieren & erledigt & erreicht & Sensor und GPIO wurden korrekt in Betrieb genommen\\
        \#18 Messung Sensoren & erledigt & erreicht & LA gab auf Sensor nichts aus dafür aber der provisorische Python Zugriff\\
    \end{tabular}
    \caption{Status der User Stories aus Sprint 1}
\end{table}

\clearpage
\subsection*{Sprint 3}
Die Planung und der Abschluss von Sprint 1 ist in diesem Kapitel aufgeführt.
\subsubsection*{Planung}
Sprint 3 enthält eine grosse User Story für das Entwerfen und Erstellen eines PCB mit einem Hardware Counter. Dazu wird noch die Datenpersistenz entwickelt.
\begin{figure}[H]
    \centering
    \includegraphics[width=.5\textwidth]{sprint3_plan.png}
    \caption{Sprintplanung für Sprint 3}
\end{figure}
\end{document}
