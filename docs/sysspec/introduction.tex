% !TEX root = SysSpec_ClockPendulumAnalyzer.tex
\section{Einführung}
		\subsection{Zweck des Dokuments}
            Dieses Dokument gibt einen Überblick auf die Umsetzung und Entscheidungen die im Laufe des Projektes getroffen wurden.
		\subsection{Zielpublikum}
            Das Zielpublikum dieses Dokuments sind weitere Entwickler und die Experten und Auftraggeber für diese Arbeit.
		\subsection{Versionierung}
			\begin{table}[h]
				\centering
				\begin{tabularx}{\textwidth}{|c|c|X|}
				\hline
				\rowcolor{shadecolor}\textbf{Version} & \textbf{Datum} & \textbf{Kommentar}\\ \hline
                V1.4 & 17.11.2017 & Kapitel ''Klassendiagramm'' und ''Umsetzung der Datenpersistenz'' erstellt\\ \hline
                V1.3 & 09.11.2017 & Systemkontext erfasst.\\ \hline
                V1.2 & 02.10.2017 & Um Kapitel Hardware erweitert \\ \hline
        		V1.1 & 05.10.2017 & Vision komplettiert \\ \hline
				V1.0 & 28.09.2017 & initial file \\ \hline
				\end{tabularx}
			\end{table}
		\subsection{Glossar}
			\begin{description}
				\item[CPA]- Clock Pendulum Analyzer
                \item[RPi3] Abkürzung für das Raspberry Pi version 3
                \item[RTC] Real Time Clock, engl. für Echtzeituhr
                \item[I2C] ein Busprotokoll für Embedded Geräte (auch als $I^2C$ bezeichnet)
                \item[SSH] Secure SHell, ein sicheres Verbindungsprotokoll
                \item[GPIO] General Purpose Input Output. Eine Reihe von Pins für In- und Output Operationen
			\end{description}