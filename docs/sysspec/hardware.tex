%!TEX root = SysSpec_ClockPendulumAnalyzer.tex
\section{Hardware Komponenten}
\subsection{IR-Sensoren}
	Alle in Frage kommenden Sensoren für die Ermittlung der Pendeldurchgänge basieren auf optischer Technologie, da Schallsensoren nicht genau genug sind oder zu nahe am Pendel platziert werden müsten. Mechanische Sensoren können aufgrund der physikalischen Effekte auf das Pendel verworfen werden.\\
	 Bei den optischen Sensoren steellt sich die Frage, ob der Sender und der Empfänger sich im gleichen Sensor befinden und ob ein Reflektor benötigt wird oder nicht.\\
	Aufgrund der unterschiedlichen Bauarten der Pendeluhren haben wir uns bezüglich für einen Sensortyp entscieden, welcher keinen Reflektor benötigt und bei dem Sender und Empfänger sehr nahe beieinander liegen. Auf diese Weise sind auch kleine Pendel messbar und es müssen keine Reflektoren angebracht werden.\\
	\textbf{TODO: Sensorbeschrieb mit Bild und Verweis auf Anhang}
\subsection{GPS Receiver}
	Ein GPS Modul wird benötigt um den Sekundentakt exakt zu synchronisieren und so keinere Ungenauigkeiten der Messfrequenz ausgleichen zu können. Weiter dient das GPS-Modul zur Ermittlung der aktuellen Absolut-Zeit.\\
	Die Positionsfunktion des GPS-Moduls wird nicht benötigt und implementiert.
\subsection{Real Time Clock}
	Die Real-Time Clock (RTC) kann als externer Frequenzgeber verwendet werden, da sie über ein 32KHz signal verfügt, welches eine Abweichung von +/- 2 ppm\footnote{Parts Per Million} auf ein \textbf{TODO: exakte Zeiteinheit für die Toleranz einfügen} verspricht.\\
	Weiter verfügt die RTC über die Möglichkeit, dem Raspberry PI die Absolutzeit zu vermitteln, wenn z.B. keine Netwerkverbindung besteht oder kein NTP\footnote{Network Time Protocol} verfügbar ist.
\subsection{Hardware Counter}
%!TEX root = SysSpec_ClockPendulumAnalyzer.tex
\subsection{Raspberry Pi}
Die ganze Software wird mittels C++ auf einem Raspberry Pi Version 3 (kurz RPi3) ausgeführt.

\subsubsection{Entscheidungsgrundlage}
Für die Umsetzung auf Computerebene wurde entschieden kein Realtime OS zu verwenden. Das führte dazu, dass eine der möglichen Lösungen auf das Raspberry Pi fiel. Das Raspberry bietet eine weitverbreitete und aktive Community. Dadurch kann bei Problemen leicht auf gute Lösungsvorschläge zugegriffen werden.\\
Um mit der Hardware auf einem guten Stand zu sein, wurde die neuste Version 3 des Raspberry Pi gewählt.

\subsubsection{Betriebssystem und Software}
Für das Betriebssystem wird ein extrem leicht gewichtetes OS verwendet. Dies verhindert dass ungewünschte daemons oder andere Programme den Betrieb verlangsamen. Es wird daher nur das allernötigste installiert. In der untenstehenden Auflistung sind alle zusätzlich installierten Programme aufgelistet. 

\begin{table}[h]
    \begin{tabular}{ll}
        ArchlinuxARM & OS für Raspberry Pi\\
        i2c-tools 3.1.2-1 & i2c tool set für Linux\\
        libconfig & C/C++ Konfiguration Datei Bibliothek\\
        libbcm2835 & Broadcom BCM 2835 c Bibliothek für Raspberry Pi\\
    \end{tabular}
    \caption{installierte Software auf dem RPi3}
\end{table}

\noindent Das Betriebssystem wird mit einem Benutzer und möglichem SSH Zugriff eingerichtet. Um per SSH auf das Raspberry zugreifen zu können muss der Host im Netzwerk 192.168.192.0/24 sein.
\begin{table}[h]
    \begin{tabular}{ll}
        Benutzer: & clockpendulum \\
        Passwort: & cpa\_admin \\
        IP: & 192.168.192.75 \\
    \end{tabular}
    \caption{Daten für den Zugriff auf das RPi3}
\end{table}
%user: clockpendulum
%pw: cpa_admin