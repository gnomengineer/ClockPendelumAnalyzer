%!TEX root = SysSpec_ClockPendulumAnalyzer.tex
\section{Ausgangslage}%TODO Problem ersichtlich machen
Eine Pendeluhr besteht grundsätzlich aus drei Komponenten: dem Energiespeicher (eine Spiralfeder
oder ein Gewicht), einer Dosiereinheit (Pendel mit Hemmung), welche die Energie portioniert an ein
Räderwerk abgibt und einer Anzeigeeinheit (das Zifferblatt).\\
Die Genauigkeit einer Pendeluhr wird wesentlich bestimmt durch das Pendel. Die volle Periode eines Sekundenpendels sollte exakt 2 Sekunden betragen; dann läuft die Uhr exakt. Eine Abweichung von 1 Millisekunde addiert sich im Laufe des Tages auf einen Fehler von ca. 43 Sekunden.\\
Die genauste, rein mechanische Pendeluhr der Geschichte, die AChF-3, wurde von Fedchenko 1959
hergestellt. Die durchschnittliche Abweichung pro Tag betrug 0.003 s.\\
Durch die genaue Messung einer Periode kann der Fehler pro Tag vorausgesagt werden. Durch (a)
Verlängern oder Verkürzen des Pendels oder durch (b) Vergrössern oder Verkleinern des
Pendelgewichts kann dann der Fehler minimiert werden. Zur Feinjustierung wird im Falle (b) meist ein
Gewicht verwendet, welches etwa in halber Pendelhöhe angebracht wird. Beispielsweise wird dazu
beim Big-Ben in London ein \glqq{}Penny\grqq{} verwendet, welcher zu einer Änderung der Zeit von 0.4 s pro Tag führt.\\
Bei Pendeluhren in Privathaushalten erfolgt die Ermittlung der Abweichung durch Beobachten der Zeitabweichung über eine gewisse Zeitspanne, z.B. eine Woche oder ein Monat, gegenüber einer Referenz. Anschliessend kann die Nachjustierung erfolgen. 

\section{Ziele}
Es soll ein Device erstellt werden, welches die Messung eines Pendels ermöglicht und so die Genauigkeit (Abweichung pro Tag) des Pendels direkt anzeigen kann.\\
Dabei sollen die Messungen in (Sub)-Mikrosekunden erfolgen.\\
Weiter sollen die Messdaten eines Pendels auch über längere Zeit beobachtet werden können.\\
Die Messdaten sollen visualisiert werden um die Auswertung der Daten zu erleichtern.

\section{Scope}
Dieses Projekt beschränkt sich auf das Erstellen von Funktionsmuster (Prototypen) inklusive Visualisierung der Messdaten.\\
Es werden ausschliesslich die folgenden Pendeltypen unterstützt:
\begin{itemize}
	\item Einfachpendel
	\item Doppelpendel
	\item Kreispendel (Typ Atmos)
\end{itemize}
Die Pendellänge muss zwischen 150 und 1000mm liegen, der Durchmesser bei Kreispendeln zwischen 50 und 300mm\\
Die Konstruktion eines Pendels ist nicht Teil dieses Projekts.

\section{Anforderungen}
\subsection{funktionale Anforderungen}
	\begin{itemize}
        \item Die Genauigkeit soll mit einer quadratischen Approximation erfolgen.
        \item Die Messung soll an einem Punkt vorgenommen werden.
        \item Als Referenzclock soll für Prototyp 1 eine 32kHz RTC\footnote{RTC = Real Time Clock} verwendet werden, für Prototyp 2 eine GPS- disziplinierte 10MHz RTC.
        \item Die Messvorrichtung darf keinerlei mechanische Eingriffe an dem zu messenden Pendel erfordern und soll hinter dem Pendel angebracht werden können.%unklar ob nicht-f oder funktional
        \item Die Messdaten müssen persistent gespeichert werden.
		\item Die Messdaten müssen in einem Webbrowser visualisiert werden. Es sollen die Browser Google Chrome, Firefox, Safari und Opera unterstützt werden.
		\item Die Messvorrichtung muss auf die Grösse des Pendels anpassbar sein, um verschieden grosse Pendel zu unterstützen
		\item Der Pendeltyp, der gemessen wird, soll vom Systembenutzer festgelegt werden können.
		\item Der Zeitbereich der angezeigten Daten soll vom Benutzer gewählt werden können.
		\item Die Anzeige soll sowohl auf Desktop-PC's / Notebooks, wie auch auf mobilen Geräten möglich sein.
        
		\item (Optional) Es soll die Temperatur gemessen werden, um einen allfälligen Einfluss auf die Pendelgenauigkeit anzeigen zu können.
        \item (Optional) Es soll die Luftfeuchtigkeit gemessen werden, um einen allfälligen Einfluss auf die Pendelgenauigkeit anzeigen zu können.
	\end{itemize}
\subsection{nicht-funktionale Anforderungen}
	\begin{itemize}
		\item Die Messgenauigkeit muss im Mikrosekundenbereich liegen.
        \item Die Messdaten müssen über einen Zeitraum von 3 Monaten verfügbar sein.
        \item Die Messdaten dürfen bei einem Stromunterbruch nicht verloren gehen.
		\item Die Repräsentation der Daten soll in grafischer und tabellarischer Form erfolgen.
        \item (Optional) Die Messgenauigkeit muss im Nanosekundenbereich liegen.
	\end{itemize}

\section{Resultate}
Es sind die Folgenden Resultate zu erstellen:
\begin{itemize}
	\item[\textbf{R1:}] Hardwaresystem für die Messung des Pendels.
	\item[\textbf{R2:}] Applikationssoftware auf dem Messsystem.
	\item[\textbf{R3:}] Applikationssoftware für die Darstellung der Messdaten.
	\item[\textbf{R4:}] Dokumentation.
\end{itemize}

\section{Business Case}
    Mit dem zu erstellenden Gerät kann die Abweichung einer Uhr viel effizienter gemessen werden. Der Benutzer braucht nicht mehr Stunden damit zu verbringen die Uhr zu beobachten und kann anhand der Messung die Uhr schneller nachjustieren.\\
    Weiter kann mit dem Gerät der Einfluss von Temperatur und Feuchtigkeit beachtet werden und somit eine qualitative Verbesserung der Einstellung erzeugt werden.