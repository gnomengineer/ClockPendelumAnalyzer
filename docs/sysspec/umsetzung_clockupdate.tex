% !TEX root=SysSpec_ClockPendulumAnalyzer
\subsection{Umsetzung des Clock Update}%TODO rework chapter clock update
Als Clock Update wird der Prozess verstanden, der die Systemzeit des Raspberry aktualisiert.

\subsubsection{Problemstellung}
Das System ist nicht zwingend an einem Internet angeschlossen. Über ein einfaches Lokales Netzwerk kann bereits mit HTTP ein Datenabruf stattfinden. Dies bedeutet, dass sich das Raspberry nicht mittels NTP Protokoll aktuell halten kann.

\subsubsection{Update der Systemzeit}
Als Ablösung für das NTP Protokoll wird das am Hardware Counterboard angehängte GPS Modul verwendet. Über den bereits im Kapitel \ref{sec:uart} beschriebenen UART Anschluss kann die genaue Zeit genommen werden.\\
Dieser Zeitwert vom GPS wird dann für das Setzen der Hardware Clock auf dem Raspberry verwendet.\\
\\
Mit einer genauen Uhrzeit und Datum im System kann ein besser Zeitstempel für den Messwert genommen werden. 

\subsubsection{Systemzeit für Messdaten}
Das System erhält einen kontinuierlichen Zeitstempel bei jedem Messdaten Sampling. Dabei wird während der Zusammensetzung von allen Informationen ein Zeitstempel von der GPS regulierten Systemzeit genommen. Unter Informationen sind Angaben zur absoluten Zeit, Uhrennamen und Datum zu verstehen.\\
\\
Der Zeitstempel ist in einer Form, die für die Datenbank geeignet ist. Näheres zum Format ist dem Kapitel \ref{sec:db_date} zu entnehmen.