%!TEX root = SysSpec_ClockPendulumAnalyzer.tex
\subsection{Raspberry Pi}
Ein Grossteil der Software wird mittels C++ auf einem \rpi\ ausgeführt.

\subsubsection{Entscheidungsgrundlage}
Für die Umsetzung auf Computerebene wurde entschieden kein Realtime OS\footnote{Operations System} zu verwenden. Eine Alternative bilden die verschiedenen Kleincomputer wie \rpi, Freedom Board, Beagle Board und viele weitere. Die Entscheidung fiel auf das \rpi. Es bietet eine weitverbreitete und aktive Community. Dadurch kann bei Problemen leicht auf gute Lösungsvorschläge zugegriffen werden.\\
Um mit der Hardware auf einem guten Stand zu sein, wurde die neuste Version 3 des Raspberry Pi gewählt.

\subsubsection{Betriebssystem und Software}
Für das Betriebssystem wird ein extrem leicht gewichtetes OS verwendet. Dies verhindert, dass unerwünschte daemons\footnote{Hintergrundprozess unter linux} oder andere Programme den Betrieb verlangsamen. Es wird daher nur das Aller-nötigste installiert. In der untenstehenden Auflistung \ref{tab:installed_sw} sind alle zusätzlich installierten Programme aufgelistet. 

\begin{table}[h]
    \begin{tabular}{ll}
        ArchlinuxARM & OS für das \rpi\\
        i2c-tools 3.1.2-1 & \iic \tablefootnote{Inter-Integrated Circuit} tool set für Linux\\
        libconfig & C/C++ Konfiguration Datei Bibliothek\\
        libbcm2835 & Broadcom BCM 2835 c Bibliothek für das \rpi\\
    \end{tabular}
    \caption{installierte Software auf dem \rpi}
    \label{tab:installed_sw}
\end{table}

\noindent Das Betriebssystem wird mit einem Benutzer und möglichem SSH\footnote{Secure Shell} Zugriff eingerichtet. Um per SSH auf das \rpi\ zugreifen zu können muss der Host im Netzwerk 192.168.192.0/24 sein. Die Benutzerangaben sind in der Tabelle unten (Tabelle \ref{tab:pi_user})
\begin{table}[h]
    \begin{tabular}{ll}
        Benutzer: & clockpendulum \\
        Passwort: & cpa\_admin \\
        IP: & 192.168.192.75 \\
    \end{tabular}
    \caption{Daten für den Zugriff auf das RPi3}
    \label{tab:pi_user}
\end{table}