%!TEX root=Projektdokumentation_ClockPendulumAnalyzer.tex
\section{Einführung}
		\subsection{Zweck des Dokuments}
            Dieses Dokument gilt als Projektdokumentation über den ganzen Zeitraum des Projektverlaufs. Es beinhaltet Zusammenfassung, Vision, Projektmanagement, System Spezifikationen und Fazit.
		\subsection{Zielpublikum}
            Das Zielpublikum dieses Dokuments sind weitere Entwickler, die Experten und Auftraggeber dieser Arbeit.
		\subsection{Versionierung}
			\begin{table}[h]
				\centering
				\begin{tabularx}{\textwidth}{|c|c|X|}
				\hline
				\rowcolor{shadecolor}\textbf{Version} & \textbf{Datum} & \textbf{Kommentar}\\ \hline
                V1.7.0 & &...\\ %TODO Versionierung ergänzen
                & 14.12.2017 & Sprint 5 Review hinzugefügt\\
                V1.6.0 & 07.12.2017 & Kapitel ''Sequenzdiagramme'' und ''Hardware Komponenten'' erweitert\\
                V1.5.1 & 01.12.2017 & Sprint 4 Review hinzugefügt\\
                V1.5.0 & 30.11.2017 & Kapitel ''Hardware Komponenten'' erstellt\\
                V1.4.0 & 17.11.2017 & Kapitel ''Klassendiagramm'' und ''Umsetzung der Datenpersistenz'' erstellt\\
                & 17.11.2017 & Sprint 3 Review hinzugefügt\\
                V1.3.0 & 09.11.2017 & Systemkontext erfasst\\ 
                V1.2.2 & 02.11.2017 & Sprint 2 Review hinzugefügt\\
                V1.2.1 & 19.10.2017 & sprint 1 Review hinzugefügt\\
                V1.2.0 & 02.10.2017 & Um Kapitel Hardware erweitert \\ 
        		V1.1.1 & 05.10.2017 & Vision komplettiert \\ 
                V1.1.1 & 29.09.2017 & organisation, rahmenplan und projektrisiken hinzugefügt\\
				V1.0.0 & 28.09.2017 & initial file \\ \hline
				\end{tabularx}
			\end{table}
		\subsection{Glossar}
			\begin{description}
				\item[CPA]- Clock Pendulum Analyzer
                \item[RPi3] Abkürzung für das Raspberry Pi version 3
                \item[RTC] Real Time Clock, engl. für Echtzeituhr
                \item[I2C] ein Busprotokoll für Embedded Geräte (auch als $I^2C$ bezeichnet)
                \item[SSH] Secure SHell, ein sicheres Verbindungsprotokoll
                \item[GPIO] General Purpose Input Output. Eine Reihe von Pins für In- und Output Operationen
			\end{description}