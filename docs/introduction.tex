%!TEX root=Projektdokumentation_ClockPendulumAnalyzer.tex
\section{Einführung}
		\subsection{Zweck des Dokuments}
            Dieses Dokument gilt als Projektdokumentation über den ganzen Zeitraum des Projektverlaufs. Es beinhaltet Zusammenfassung, Vision, Projektmanagement, System Spezifikationen und Fazit.
		\subsection{Zielpublikum}
            Das Zielpublikum dieses Dokuments sind weitere Entwickler, die Experten und Auftraggeber dieser Arbeit.
		\subsection{Versionierung}
			\begin{table}[h]
				\centering
				\begin{tabularx}{\textwidth}{|c|c|X|}
				\hline
				\rowcolor{shadecolor}\textbf{Version} & \textbf{Datum} & \textbf{Kommentar}\\ \hline
                V1.8.0 & 03.01.2018 & Überarbeitung der Kapitel ''Umsetzung des Zählers'',''Umsetzung Sensor'' und ''Umsetzung UI''\\
                V1.7.1 & 21.12.2017 & Allgemeine Fehlerkorrektur\\
                V1.7.0 & 15.12.2017 & Kapitel ''Umsetzung Sensor'' und ''Umsetzung UART/\iic'' überarbeitet\\ 
                & 14.12.2017 & Sprint 5 Review hinzugefügt\\
                V1.6.0 & 07.12.2017 & Kapitel ''Sequenzdiagramme'' und ''Hardware Komponenten'' erweitert\\
                V1.5.1 & 01.12.2017 & Sprint 4 Review hinzugefügt\\
                V1.5.0 & 30.11.2017 & Kapitel ''Hardware Komponenten'' erstellt\\
                V1.4.0 & 17.11.2017 & Kapitel ''Klassendiagramm'' und ''Umsetzung der Datenpersistenz'' erstellt\\
                & 17.11.2017 & Sprint 3 Review hinzugefügt\\
                V1.3.0 & 09.11.2017 & Systemkontext erfasst\\ 
                V1.2.2 & 02.11.2017 & Sprint 2 Review hinzugefügt\\
                V1.2.1 & 19.10.2017 & sprint 1 Review hinzugefügt\\
                V1.2.0 & 02.10.2017 & Um Kapitel Hardware erweitert \\ 
        		V1.1.1 & 05.10.2017 & Vision komplettiert \\ 
                V1.1.1 & 29.09.2017 & organisation, rahmenplan und projektrisiken hinzugefügt\\
				V1.0.0 & 28.09.2017 & initial file \\ \hline
				\end{tabularx}
			\end{table}
        
        \clearpage
		\subsection{Glossar}
			\begin{description}
				\item[CPA -] Clock Pendulum Analyzer
                \item[Target -] Bezeichnet den Computer auf welchem am Ende die Software laufen soll
                \item[RTC -] Real Time Clock, engl. für Echtzeituhr
                \item[I2C -] Inter-Integrated Circuit, Busprotokoll für Embedded Geräte (auch als \iic bezeichnet)
                \item[SSH -] Secure SHell, ein sicheres Verbindungsprotokoll
                \item[GPIO -] General Purpose Input Output. Eine Reihe von Pins für In- und Output Operationen
                \item[SDA -] Serial DAta, Datenleitung des \iic
                \item[SCL -] Serial CLock, Taktleitung des \iic
                \item[GPS -] Global Positioning System, globales Navigationssatellitensystem
                \item[PPS -] Pulse Per Second, engl. Bezeichnung für den Puls pro Sekunde
                \item[PPM -] Parts Per Million, engl. Bezeichnung für Stück pro Million
                \item[NTP -] Network Time Protocol, Protokoll zur Zeitmessung auf Computern
                \item[IC -] Integrated Circuit, engl. Bezeichnung für einen eingebauten Schaltkreis
                \item[OS -] Operation System, engl. Bezeichnung für Betriebssystem
                \item[ARM -] Advanced RISC Machines, weitverbreiteter Computerchip
                \item[RISC -] Reduced Instruction Set Computing, Bezeichnung für einen Computerprozessor mit reduziertem Befehlssatz
                \item[RAII -] Resource Aquisition is Initialization, ein verbreitetes Programmierprinzip der Resourcenverwaltung
                \item[IR -] Infrarotstrahlung,  elektromagnetische Wellen im Bereich 1mm - 780mm
                \item[PCB -] Printed Circuit Board, engl. Bezeichnung für eine Leiterplatte
                \item[SD-Karte -] Secure Digital Memory Card, engl. Bezeichnung ein digitales Speichermedium
                \item[SQLite -] Datenbanktechnologie für relationale Datenbanken
                \item[MySQL -] Datenbanktechnologie für relationale Datenbanken
                \item[Regular Expression -] regulärer Ausdruck zum Beschreiben von Zeichenketten
                \item[SQL -] Structured Query Language, Abfragesprache für Datenbanken wie MySQL und SQLite
                \item[HTTP -] Hyper Text Transfer Protokoll
                \item[REST -] Representational State Transfer, eine auf HTTP basierende Kommunikationsmöglichkeit
                \item[JSON -] JavaScript Object Notation, kompaktes Datenformat für Datenaustausch
			\end{description}