%!TEX root=Projektdokumentation_ClockPendulumAnalyzer.tex
\section{Management Summary}
Eine grosse Herausforderung bei Zeitmessungen ist die Genauigkeit. So können zum Beispiel Pendeluhren durch ein fehlerhaft eingestelltes Pendel nach geringer Zeit ungenau werden. Durch Beobachten der Pendeluhr und Kalibrieren des Pendels, kann die Uhr wieder gestimmt werden. Dieser Prozess ist mühsam und zeitintensiv.\\
\\
In dieser Arbeit ''\documenttitle'' wird ein Gerät entwickelt, dass es dem Anwender erleichtern soll, die Genauigkeit seiner Pendeluhr fest zu stellen.\\
Das entwickelte System schafft es auf 1-10 Mikrosekunden genau, die Zeitdifferenz zu berechnen und diese dem Anwender mitzuteilen.\\
\\
Im Rahmen dieser Arbeit wurde ein kleines Hardware Board entwickelt, welches mit Hilfe eines Infrarot-Sensors die Zeit eines Pendeldurchgangs misst.\\
Die gemessene Zeit wird anhand einer genauen GPS-Zeitsekunde verglichen und somit die Abweichung festgestellt.\\
\\
Die Messdaten werden über Monate hinweg in einer Datenbank gespeichert.
Dem Anwender werden diese Daten über eine Weboberfläche aufbereitet und dargestellt.\\
Das ganze System wurde möglichst modular gehalten. Es kann ohne grössere Probleme an beliebige Pendeluhren angepasst werden.\\
\\
Das Ergebnis ist ein sehr genaues Gerät zur Messung der Zeitabweichung. 
Jedoch sind einige Module nur teilweise umgesetzt und auf der Weboberfläche ist nur das Nötigste zu sehen. 
Durch den Aufbau und die Planung des ganzen Systems kann aber ohne grossen Aufwand weitergearbeitet werden. 
So besteht die Möglichkeit eine neue Benutzeroberfläche zu erstellen, welche als eigenständige Applikation laufen kann. Die Daten erhält diese durch die bereits definierte Schnittstelle.

%Arbeit, Ziele und Querverbindungen
%Wertung, Ausblick und Beantwortung der Hypothese