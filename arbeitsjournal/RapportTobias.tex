%\newcommand{\header}{\textbf{Work Item}&\textbf{Datum}&\textbf{Zeitaufwand (in Std)}\\\toprule}
%\newcommand{\footer}{\midrule\textbf{Total Arbeitsstunden}&-&\textbf{157.00}\\\midrule\bottomrule}
    \begin{tabular}{ll}
        Entwickler: & Tobias Kreienbühl \\
        Arbeitsperiode: & 18.09.2017 - 22.12.2017\\
    \end{tabular}

	\begin{longtable}{p{9cm}|p{2cm}|c}
		\header
		\textbf{Beschreibung} 	& \textbf{Datum} 	& \textbf{Zeitaufwand}\\
		Recherche Sensoren 		& 27.09.2017 		& 1h 30min\\\midrule
            Erstellen Grobkonzept 	& 28.09.2017 		& 3h 15min\\\midrule
            Erweiterung Grobkonzept & 01.10.2017 		& 4h\\\midrule
            Zwischenbesprechung und Planung Layout der Hardware	& 05.10.2017	& 5h\\\midrule
            Defintive Entscheidung Hardware		& 12.10.2017 & 4h\\\midrule
            Beschaffung Hardware und Material für Testgehäuse  & 14.10.2017 & 4h 15min \\\midrule
		Erstellen Testgehäuse  & 15.10.2017 & 4h\\\midrule
		Gemäss erster gedachter Schaltung, Erstellen eines Prototypenboardes für den Sensor & 18.10.2017 & 3h 30min\\ \midrule
		Annahme RTC und GPS-Modul & 19.10.2017 & 30min \\\midrule
		Versuch1: Sensor an Raspberry Pi anhängen und testen -> vermeindlicher Erfolg & 19.10.2017 & 4h 30min \\ \midrule
		Analyse Datenblätter und anlöten erster Pins zum Test der RTC und GPS & 26.10.2017 & 7h 30min\\ \midrule
		Beschaffen Logic-Analyser und kurze Instruktion & 26.10.2017 & 30min \\\midrule
		Weitere Recherche RTC & 28.10.2017 & 2h \\\midrule
		Auswerten RTC 32kHz Frequenz mit Hilfe Logic Analyser und Versuchsboard & 02.11.2017 & 5h 30min \\\midrule
		Recherche Hardware-Counter, Evaluieren von möglichen IC's & 02.11.2017 & 2h\\ \midrule
		Beschaffen eines TinyK20, anlöten Footer & 03.11.2017 & 3h 30min \\\midrule
		Inbetriebnahme TinyK20 mit Kinetis Design Studio und Debugger & 04.11.2017 & 3h \\ \midrule
		Verbinden und Testen GPS mit TinyK20, Sysnchonisation testen. & 9.11.2017 & 4h 30min\\\midrule
		Versuch, Sensor anzuhängen, fehlschlag & 16.11.2017 & 4h 30min\\\midrule
		Prototypen-Board mit Steckplätzen für GPS, RTC und Tiny erstellt & 22.11.2017 & 2h 30min \\ \midrule
		Weitere Versuche mit Sensor und Tiny, GPS-Verbindung mit Interrupt erfolgreich & 23.11.2017 & 5h 30min\\ \midrule
		Besprechung & 23.11.2017 & 30min \\ \midrule
		Layout PCB für Sensor nach erstem Schema & 27.11.2017 & 1h 30min \\\midrule
		Verbindungen fix auf Prototypen-Board anbringen & 29.11.2017 & 2h 30min \\\midrule	
		Ausbau Software für TinyK20 & 30.11.2017 & 4h 30min\\\midrule
		Entgegennehmen und Löten PCB's für Sensoren & 6.12.2017 & 2h 30min\\\midrule
		Dokumentation & 07.12.2017 & 8h 30min \\\midrule
		Dokumentation & 10.12.2017 & 6h 30min \\\midrule
		Ermitteln Fehler am Sensor & 13.12.2017 & 3h 30min\\\midrule
		Erstellen neues Schema, neues PCB in Auftrag geben & 13.12.2017 & 2h 30min\\\midrule
		Dokumentation & 14.12.2017 & 4h 30min\\\midrule
		Hardwareboard erweitern & 20.12.2017 & 5h 30min \\ \midrule
		Neues PCB bestücken und testen, Softwareerweiterung Mokrocontroller & 21.12.2017 & 9h 30min  \\ \midrule
		Steckeranschluss ändern und Kabelkonfektion & 22.12.2017 & 3h 30min  \\ \midrule
		Gehäusekonstruktion & 26.12.2017 & 6h  \\ \midrule
		Gehäusefertigung & 27.12.2017 & 8h 30min  \\ \midrule
		Gehäusefertigung & 28.12.2017 & 3h 30min  \\ \midrule
		Anschlüsse für Bauteile am Gehäuse, neues PCB für Sensor bestücken und testen. Gesamttest Hardware & 29.12.2017 & 7h 45min  \\ \midrule
		Gehäuse fertig montieren, lackieren, auskleiden, ölen & 30.12.2017 & 6h 30min  \\ \midrule
		Montage der elektronischen Bauteile im Gehäuse, Test der Hardware, Auskleiden mit Schallschutz & 31.12.2017 & 5h \\ \midrule
		Systemtest mit Uhr, Dokumentation & 02.01.2018 & 8h 30min \\ \midrule
		Dokumentation & 03.01.2018 & 4h 30min \\ \midrule
		Dokumentation & 04.01.2018 & 6h 45min \\ \midrule
		Dokumentation & 05.01.2018 & 4h \\ \midrule
		%TOTAL 			
		\textbf{Total Arbeitsstunden}
		&
            -
            &
		\textbf{194h}\\\midrule\bottomrule
	\end{longtable}

\noindent Es stehen noch einzelne Arbeitspakete aus, die nach Abgabe des Arbeitsjournal anfallen. Diese Aufgaben werden in der untenstehenden Auflistung geschätzt.\\
\\
Ausstehende Arbeiten:\\
\begin{tabular}{p{9cm}|c}
    \textbf{Work Item}     & \textbf{Zeitaufwand (in Std)} \\\hline
    Präsentation entwerfen & 8\\
    Präsentation vorbereiten & 8 \\
    anfallende Bug Fixes   & 2\\ \midrule
    \textbf{Total geschätzte Stunden} & \textbf{10}\\ \midrule\bottomrule
\end{tabular}